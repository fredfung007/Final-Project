\section{Methods}
%主要写我们做了哪些事情,各个算法的参数都是怎么选择的。
We tested Adaboost and Deepboost on the UCI dataset, ionosphere\cite{Lichman2013}. We randomly split the data to two parts; 80\% for the training data and 20\% for testing.
\subsection{Adaboost}

\subsection{Deepboost}
Following the work from Cortes et al. \cite{cortes2014deep}, we use the $H_1^{stumps}$ as the base classifier for Deepboost.
The Rademacher complexity of $H_1^{stumps}$ can be bounded by its growth function.
It is trivial to see that
\[\Pi_{H_1^{stumps}}(m) \leq 2md,\]
since there are $2m$ distinct threshold functions for each dimension with $m$ points. Therefore,
\[\mathfrak{R}_m (H_1^{stumps})\leq \sqrt{\frac{2\log (2md)}{m}}.\]

By now, we have the notation from Deepboost
\[\Lambda_j=\lambda\cdot\mathfrak{R}_m (H_1^{stumps}) +\beta.\]
where we conducted experiments for $\lambda \in \{10^{-i}:i=3,\cdots, 7\}$ and $\beta \in \{10^{-i}:i=3,\cdots, 7\}$ as well, and optimize the training error on these experiments.

We optimize the parameter by minimizing the 10-fold cross validation, and then measure the error by testing data.

In all of our experiments, the number of iterations was set to 50. We also test the result for 100 rounds, but the test error remains basically the same.
As we shall see, in some experiments the training error decreases vastly and reaches to zero after 40 iterations.
