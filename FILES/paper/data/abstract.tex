\begin{abstract}
Organizing taxis in a metropolis has significant social and economic benefits, for instance in reducing the waiting time of passengers and improving the efficiency of taxis. Even though rearranging taxi resources with respect to their own characteristics can be illuminating, taxi arrangement is not well studied.

In this paper, prediction problem of urban structure is defined and an item-level multi-factor prediction method is proposed. Taxi's influence on a happening of travel is revealed in a more accurate way. In this item-level approach, both the taxi characteristic and location influences are considered, compared to previous works. Taxis are clustered using 5 major attributes, while locations are divided into areas adjacent to each other. Additionally, time variation is also considered in the proposed model. The problem is then formulated as compute the probability of a specific taxi gets a travel (customer) at a specific location in certain time period of a day. The underling mathematical model is a tensor optimization model, which is later processed. The advantage of the proposed method (PHF-MF) was evaluated by computing the Root Mean Square Error (RMSE) with a comparison for two other methods, using past taxi travel records gathered in Beijing.
\end{abstract}