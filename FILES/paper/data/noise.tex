\section{Realistic Noise}
The impact of noise in the performance of an algorithm is regarded as the robustness.
In Diettterich’s work\cite{dietterich2000experimental}, multiple levels of noise are added to sample dataset to test the robustness of Adaboost.
However, the noise was introduced by reverting labels in training data randomly without replacement with a fraction $r$.
This makes the noise in the training dataset unrealistic since the noise is very arbitrary (in figure \ref{fig:uniform}).
Our idea is to infer the noise distribution through AdaBoost.


\begin{figure}
\begin{center}
\resizebox {\columnwidth} {!} {

\begin{tikzpicture}
\begin{axis}[%
    scale only axis,
    xmin=0,
    axis x line*=bottom,
    axis y line*=left
]
\addplot[%
    scatter=true,
    only marks,
    mark=*,
    color=blue,
    point meta=explicit symbolic,
    scatter/@pre marker code/.style={/tikz/mark size=\pgfplotspointmeta/50},
    scatter/@post marker code/.style={}
] table [meta index=2, x index=1, y index=0] {table/uniform.txt};
\addplot[%
    scatter=true,
    only marks,
    mark=*,
    color=red,
    point meta=explicit symbolic,
    scatter/@pre marker code/.style={/tikz/mark size=\pgfplotspointmeta/50},
    scatter/@post marker code/.style={}
] table [meta index=2, x index=1, y index=0] {table/uniformr.txt};
\addplot[dashed]{5};
\end{axis}
\end{tikzpicture}%

}
\caption{Drawing noise according to uniform distribution}
\label{adaiono}
\end{center}
\end{figure}



\begin{figure}
\begin{center}
\resizebox {\columnwidth} {!} {

\begin{tikzpicture}
\begin{axis}[%
    scale only axis,
    xmin=0,
    axis x line*=bottom,
    axis y line*=left
]
\addplot[%
    scatter=true,
    only marks,
    mark=*,
    color=blue,
    point meta=explicit symbolic,
    scatter/@pre marker code/.style={/tikz/mark size=\pgfplotspointmeta/50},
    scatter/@post marker code/.style={}
] table [meta index=2, x index=1, y index=0] {table/margindist.txt};
\addplot[%
    scatter=true,
    only marks,
    mark=*,
    color=red,
    point meta=explicit symbolic,
    scatter/@pre marker code/.style={/tikz/mark size=\pgfplotspointmeta/50},
    scatter/@post marker code/.style={}
] table [meta index=2, x index=1, y index=0] {table/margindistr.txt};
\addplot[dashed]{5};
\end{axis}
\end{tikzpicture}%

}
\caption{Drawing noise according to margin distribution}
\label{adaiono}
\end{center}
\end{figure}


\subsection{Margin}
Following the margin theory in Adaboost, we have the margin defined as
\[\rho(x) = y \frac{ \mathbf{\alpha} \cdot \mathbf{h}(x) }{ \| \mathbf{\alpha}\|_2 }.\]
The distribution maintained after $t$ rounds by Adaboost is
\[D_{t+1}(i) = \frac{\exp( -y_i \mathbf{\alpha}_t \cdot \mathbf{h}_t (x_i) ) }{m \prod_{s=1}^t Z_s}.\]

Intuitively, noise happens when features are not distinguishing. In other words, real world noise happens with high probability when the point has small margin (in figure \ref{fig:margindist}).
It it trivial to see that $D_{t+1}(i)$ has a negative correlation with margin $\rho$.
Therefore, since we are introducing noise with the distribution after $t$ iterations of Adaboost with some stopping criteria,
the samples whose margin are relatively small, i.e. that are closer to the hyperplane in our experiment later and
are difficult to classify by boosting, will have a higher distribution of noise.
This makes the noise more realistic than the uniform distribution over the sample in a sense that
noise are typically distributed at the samples that has a small margin and are difficult to classify by boosting.

\subsection{Distribution of Realistic Noise}
However, in actual datasets noise are not distributed with a uniform fashion across the entire dataset.
In Xingquan et al.’s work\cite{zhu2003eliminating}, a general method to eliminate noise from training data is introduced.
In this algorithm, noise identification is based on the majority and non-objection schemes,
which is founded on the assumptions that noise is distributed according to the distribution of empirical errors.
The denser the classification errors, the denser the noise in training set.
More generally, realistic noise distribution should not be uniform over the entire training dataset, but with a relation to the classification error.

Following the definition of Adaboost, the distribution $D_t(i)$ updated during each iteration of Adaboost is a perfect simulation of the training error distribution after round $t$.
Since $D_t(i)$ is updated with according to the loss function, the distribution would be updated with a higher level where empirical errors are denser.
Therefore, the distribution from Adaboost after $T$ rounds(finished) would be suitable for introducing realistic noise.


